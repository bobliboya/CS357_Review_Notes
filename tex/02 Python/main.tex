\documentclass[12pt]{article}
\usepackage{fullpage,geometry,amsmath,hyperref,graphicx,xcolor,amssymb,array,enumitem,minted}
\usepackage{indentfirst}
\usepackage[version=4]{mhchem}
\newcommand\norm[1]{\left\lVert#1\right\rVert}
\newcommand\normx[1]{\left\Vert#1\right\Vert}
\geometry{letterpaper,left=2.54cm,right=2.54cm,top=2.54cm,bottom=2.54cm}

\begin{document}
\begin{center}\Large\bf 
CS 357 - 02 Python\\
\end{center}
\begin{center}
Boyang Li (boyangl3)
\end{center}

\medskip
\noindent \textbf{Data types:}
    \begin{minted}{python}
        a = 2           # a is int
        b = 3.0         # b is float
        c = a + b       # c is float
        d = 2 * a       # d is int
    \end{minted}

\medskip
\noindent \textbf{Names, values and modifying an object:}
    \begin{minted}{python}
        a = [1, 2, 3]       # define a as a Python list
        b = a               # b = [1, 2, 3]
        b.append(4)         # b = [1, 2, 3, 4]
        print(a)            # [1, 2, 3, 4]
    \end{minted} 

\medskip
\noindent \textbf{Memory management:}
    \begin{minted}{python}
        a = [1, 2, 3]
        b = [1, 2, 3]
        print("IS", a is b)       # "is" is used to compare ids; False
        print("EQUAL", a == b)    # "==" is used to compare values; True
        
        c = [1, 2, 3]
        d = a                     # Both "is" and "==" is True
    \end{minted}

\medskip
\noindent \textbf{Mutable and immutable objects:}
    \begin{minted}{python}
        # Mutable objects: can be modified after created
        # Example: Python lists
        myList = [3, 5, 7]
        myList[0] = 1           # myList = [1, 5, 7]

        # Tuples and strings are immuatble objects
        # the operations would result in a type error
        myTuple = (3, 5, 7)
        myTuple[0] = 1
        myString = "357"
        myString[0] = '1'
    \end{minted}

\newpage
\noindent \textbf{List operaitons:}
    \begin{minted}{python}
        a = [1, 2, 3]
        b = a
        a is b                  # True

        a = a + [4]             # a = [1, 2, 3, 4] and b = [1, 2, 3]
        a is b                  # False

        # Instead, take following operation:
        a += [4]                # a = b = [1, 2, 3, 4]
        a is b                  # True
    \end{minted}

\medskip
\noindent \textbf{Advanced naming:}
    \begin{minted}{python}
        fruit = 'apple'

        lunch = []
        lunch.append(fruit)       # lunch = ['apple']
    
        dinner = lunch            
        # dinner = ['apple'], lunch = ['apple']
        
        dinner.append('fish')    
        # dinner = ['apple', 'fish'], lunch = ['apple', 'fish']
    
        fruit = 'pear'            
    
        meals = [fruit, lunch, dinner] 
        # meals = ['pear', ['apple', 'fish'], ['apple', 'fish']]
    \end{minted}

\medskip
\noindent \textbf{Indexing:}
    \begin{minted}{python}
        # Given a list a, the formatting indexing: a[i:j:k]
        # i - starting index; j - ending index (exclusive); k - step size

        a = [0, 1, 2, 3, 4, 5, 6, 7, 8, 9]
        a[1::2][::-1]
        # output = [9, 7, 5, 3, 1]
        # First bracket starting from index 1 to the end, step size 2
        # Second bracket with step size -1, reverse the index
    \end{minted}

\newpage
\noindent \textbf{Control flow:}
    \begin{minted}{python}
        # Traditional way
        mylist = []
        for i in range(50):
          if i % 7 == 0:
            mylist.append(i**2)

        # mylist = [0, 49, 196, 441, 784, 1225, 1764, 2401]

        # Enhanced way
        mylist2 = [i**2 for i in range(50) if i % 7 == 0]
        
        # mylist2 has the same value as mylist
    \end{minted}

\medskip
\noindent \textbf{Function scope:}
    \begin{minted}{python}
        def add_minor(person):
            person.append('math')
    
        def switch_majors(person):
            person = ['physics']
            person.append('economics')
            
            # here, the person variable has a local scope, 
            # meaning its value/update only happens within the function, 
            # and is "ignored" outside of it
            
        # all variables created out here have a global scope
        John = ['computer_science']
        Tim = John
        add_minor(Tim)
        switch_majors(John)
        print(John, Tim)
        # ['computer_science', 'math'] ['computer_science', 'math']
    \end{minted}

\newpage
\noindent \textbf{Dictionary:}
    \begin{minted}{Python}
        # definition of a dictionary
        myDict = {
            "number": 357,
            "major": "computer science",
            "credit hours": 3
        }
        
        # dictionaries are mutable, thus can be modified,
        # either modifying an existing value or creating a new key
        myDict["major"] = "math"
        myDict["level"] = "undergrad"
        
        # looping through the keys or the values over the dictionary
        for x in myDict:
          print(x)
          print(myDict[x])
        
        for x in myDict.values():
          print(x)
          
        # there are multiple ways to remove an existing entry
        myDict.pop("level")
        
        # copy an existing dictionary into a new reference
        anotherDict = myDict.copy()
    \end{minted}

\medskip
\noindent \textbf{Type annotations:}
    \begin{minted} {Python}
        # simple variable declaration (the type we're used to)
        a = 5
        
        # type annotated variable declaration
        a: int = 5
        
        # simple function definition (the type we're used to)
        def add_two_numbers(a, b):
          return a + b
        
        # type annotated function definition (both inputs and output)
        def add_two_numbers(a: int, b: int) -> int:
          return a + b
    \end{minted}

\newpage
\noindent \textbf{Numpy array initialization and operation:}
    \begin{minted}{Python}
        import numpy as np
        # 2d array of zeros, shape 2 x 2
        np.zeros((2, 2))
        
        # 2d array of ones, shape 2 x 2
        np.ones((2, 2))
        
        # array of numbers evenly spaced between 2 and 3 (4 entries)
        np.linspace(2, 3, 4)
        #array([2, 2.333, 2.667, 3])
        
        # 2d array of random numbers between 0 and 1, shape 2 x 2
        np.random.rand(2, 2)
        
        # empty 2 x 2 2d array
        np.empty((2, 2))

        a = np.zeros((2, 2))
        print(a.shape)        # will return (2, 2)
        print(a.dtype)        # returns float
        a = a.astype(int)
        print(a.dtype)        # returns int
        b = a.copy()          # b is deep copy of a
    \end{minted}

\medskip
\noindent \textbf{Numpy array indexing and slicing:}
    \begin{minted}{Python}
        import numpy as np
        a = np.array([3, 7, 9, 10, 3, 5])
        b = np.array([[1, 2, 3], [4, 5, 6]])
        
        print(a[2])     # prints 9
        print(b[0, 0])  # prints 1
        
        # slicing examples for both 1d and 2d NumPy arrays
        # (for 2d arrays we specify both the row and col)
        print(a[1:3])     # prints [7, 9]
        print(b[0:1, 2])  # prints [3]
        
        # If we leave the row/col index empty or use a colon(:)
        # then we're saying that we select the entire row/col
        # this assumes the starting index of the row is 0,
        # so we select the entire first row, prints [[1, 2, 3]]
        print(b[:1])      
    \end{minted}

\newpage
\noindent \textbf{Array manipulation:}
    \begin{minted}{Python}
        import numpy as np
        a = np.array([3, 7, 9, 10, 3, 5])
        b = np.array([[1, 2, 3], [4, 5, 6]])
        
        b = np.reshape(b, (6, 1))       # Make b as 1d array
        a = np.reshape(a, (3, 2))       # Make a as 2d array
        
        a = a.flatten()                 # Flatten a directly
        
        a_transpose = a.T               # Transpose
    \end{minted}

\medskip
\noindent \textbf{Array mathematics and linear algebra:}
    \begin{minted}{Python}
        import numpy as np
        import numpy.linalg as la
        
        a = np.array([[8, 9]])
        b = np.array([[1, 2, 3], [4, 5, 6]]) 
        
        # The most important operation you'll do is matrix multiplication
        # we can do this easily in 2 ways (both are the same)
        c = np.dot(a, b)
        c = a @ b
        
        # We can do a lot of other operations 
        d = np.sin(a)       # applies to every entry in a
        e = np.cos(a)       # applies to every entry in a
        f = np.exp(a)       # applies to every entry in a
        
        g = np.sum(a)       # g is a float
        h = np.mean(a)      # h is a float
        i = np.min(a)       # i is a float

        A = np.array([1, 2], [3, 4])
        b = np.array([5, 6])
        
        matrix_inv = la.inv(A)          # inverse matrix
        
        eigval, eigvec = la.eig(A)      # eigenvalues and eigenvectors
        
        vec_norm = la.norm(A)           # calculate norm of A
        vec_norm_3 = la.norm(A, 3)      # calculate 3-norm of A
        
        x = la.solve(A, b)              # solve lienar system Ax = b
    \end{minted}

\newpage
\noindent \textbf{Random numbers:}
    \begin{minted}{Python}
        import numpy as np
    
        a = np.random.rand(3, 2)
        # 3 x 2 array of random numbers 0 to 1
        
        b = np.random.randint(100)
        # random int from 0 to 100
        x = np.random.randint(100, size=(5))
        # array of size 5 with random number 0 to 100
        x = np.random.uniform(a, b)
        # random number uniformly distributed between a and b
        
    
        c = np.random.choice([1, 2, 3, 4])
        # will randomly return one of the values within the array
    \end{minted}

\medskip
\noindent \textbf{Broadcasting:}
    \begin{minted}{Python}
        import numpy as np
        
        A = np.array([[1, 2, 3, 4, 5]])
        B = np.array([
            [10, 20, 30, 40, 50],
            [60, 70, 80, 90, 100],
            [110, 120, 130, 140, 150],
            [160, 170, 180, 190, 200]
        ])
        
        C = B + A
        
        print(C)
        # C = np.array([[ 11,  22,  33,  44,  55],
        #        [ 61,  72,  83,  94, 105],
        #        [111, 122, 133, 144, 155],
        #        [161, 172, 183, 194, 205]])
    \end{minted}

\medskip
\noindent \textbf{Packages will be useful in CS 357:}
    \begin{enumerate}
        \item \textbf{NumPy:} \href{https://numpy.org/doc/stable/reference}{https://numpy.org/doc/stable/reference}
        \item \textbf{SciPy:} \href{https://docs.scipy.org/doc/scipy/reference/}{https://docs.scipy.org/doc/scipy/reference/}
        \item \textbf{SymPy:} \href{https://docs.sympy.org/latest/reference}{https://docs.sympy.org/latest/reference}
        \item \textbf{matplotlib:} \href{https://matplotlib.org/3.5.3/api}{https://matplotlib.org/3.5.3/api}
        \item \textbf{Pandas:} \href{https://pandas.pydata.org/docs/reference}{https://pandas.pydata.org/docs/reference}
    \end{enumerate}
\end{document}